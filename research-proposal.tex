\documentclass{article}
\begin{document}

\title{AN INVESTIGATION INTO THE FEASIBILITY OF A ACTIVE ELECTRODE
EEG DATA ACQUISITION SYSTEM\\
\\
Research proposal: MEng\\
\\
Department of Electrical and Electronic Engineering\\
University of Pretoria}

\author{Thys Meintjes}
	
appointment confirmation

\maketitle

\begin{verbatim}
(Signature) 
M.J Meintjes  Feb 20, 2000


(Signature) 
Prof. P.J Cilliers
		    

(Signature)
Postgraduate Committee 
\end{verbatim}


\section{Introduction.}

A active--electrode EEG data acquisition system will be designed and
implemented. The system will be able to match or better the signal
quality of traditional systems within the constraints set by human
ergonomics and a predefined minimum system performance specification.

Traditional techniques of measuring EEG signals involve lengthy and
inconvenient scalp surface preparations before EEG signal measurements
are attempted. Inadequate scalp preparation when using conventional
EEG electrodes frequently results in low quality or unusable
data. 

The inconvenience inherent in the conventional process of EEG signal
acquisition makes it unsuitable for a wide range of applications where
EEG derived data may be applied with good effect. A user--friendly EEG
system that does not suffer from the tradition EEG acquisition problems
may be effectively applied in these application fields.

Application fields includes various uses in rehabilitation engineering
and interactive computer systems interfaces. An EEG-based human --
machine interface may provide a new communication and control modality
for people with severe motor disabilities. Replacing tradition input
mechanisms like keyboards, mice and touch screens opens up new avenues
in human computer interface design.


\section{Overview of current literature.}
Current literature predominantly concentrates on the application and
data interpretation arenas [1][2][4][5] of EEG data. Few sources
mention the specifics of data acquisition hardware employed in these
investigations. The literature does suggest various applications that
can be useful when testing the system in a chronic use scenario.
Polikoff [1] suggests using the P300 as a base calibration measurement
in "Toward a P300-based Computer Interface". Work done by Kuno [3] in
"A Study toward EEG Applications: Event -related Potential from
Posterior Parietal Cortex" suggests that Posterior Pariental Cortex
signals and EOG's might be used as a calibration indicator useful
during the evaluation of system modules.

The majority of work mentioned in the literature concerning electrode
design and fabrication concerns micro -- and invasive electrodes.
Work done by Pallas-Areny et al. in "A Improved buffer for Bioelectric
Signals" [9] discusses the use of FET operational amplifiers as active
electrodes for the acquisition of bioelectric signals. This work along
with the work done by Suzushi Nishimura et.al in [8] will form the
basis of the proposed investigation.


\section{Current state of research being done in the field.}
Current research in EEG data acquisition systems concentrate on the
interpretation of extracted signals . Various mathematical models are
being developed to map measured signals to positions and events in the
brain. Where work is being done on the acquisition side it is
concentrated on VLSI and embedded circuitry. There more activity in
the commercial sector specifically in the information technology and
gaming industries. Technical information from companies doing work in
this arena is scarce and usually protected by international copyright
law.

Available literature discussing the use of active electrodes for
bioelectric signal acquisition concentrates on EMG extraction. Mention
are made of the possibility of applying active electrode techniques to
EEG signal acquisition but no application implementing this suggestion
has been found in the literature to date.

\section{Objective}
The goal of this project is to investigate and demonstrate the
feasibility of a active electrode EEG data acquisition system. A two
channel EEG data acquisition and processing system will be designed
and implemented to this end.

The system will be shown to match or better the signal quality of
traditional systems within the constraints set by human ergonomics and
a predefined minimum system performance specification. 

The primary goal of the system is to achieve the highest possible
degree of usability within the lowest possible noise
budget. Quantitative estimates of acceptable noise levels will be used
to calculate the maximum level of permissible system noise. This noise
budget will be primarily spent on the signal acquisition module in an
effort to enhance the level of system usability.


\section{Proposed research}
The system design will be logically and physically separated into five
independent functional modules. A signal acquisition model containing
a set of active electrodes on a adjustable head--band. A low--level
signal processing module responsible for EEG signal extraction and
amplification from the signal acquisition module. A signal conversion
module capable of digitizing the EEG signal. A high level signal
processing module that allows for specialized spectral processing and
feature extraction sub-modules. And a signal display module which can
display a number of signals extracted by the high--level signal
processing module.

Each module will be designed and implemented with the elimination of
environmental and system noise as primary goal. Various techniques
will be employed to prevent the degradation of EEG signals by both
external interference and internal noise sources within each system
module.


\section{Contribution}
Your contribution in perspective with current knowledge in the field. 

Current procedures in EEG measurements dictates the use of
inconvenient signal acquisition techniques. This work will demonstrate
the feasibility of using active electrode technology [8] and computer
based digital signal processing to produce EEG data of equal or better
quality.

The use of more user-friendly EEG data acquisition systems will
contribute to the advancement of human-computer interaction
technologies.

\begin{verbatim}
References 

1. Polikoff, J., Bunnell, H.T., and Borkowski, W. (1995). "Toward a
   P300-based Computer Interface." Proceedings of the RESNA '95 Annual
   Conference, RESNAPRESS, Arlington Va. pp 178-180. [html] [ps]

2. Makeig S, Enghoff S, Jung T-P, and Sejnowski TJ, "A Natural Basis
   for Efficient Brain-Actuated Control", IEEE Trans Rehab Eng, in press.

3. Kuno, Y., Yagi, T., Uchikawa, Y., A Study toward EEG Applications:
   Event -related Potential from Posterior Parietal Cortex, Proc.19th
   Annual International Conference of the IEEE Engineering in Medicine
   and Biology Society, 1551-1553, 1997.

4. Funase, A., Yagi, T., Kuno, Y., Uchikawa, Y., Fundamental research for
   EEG interface: EEG and auditory/visual stimulus during eye-movements,
   Proc. of 1999 IEEE Int. Conf. on systems, Man and Cybernetics, Vol.II,
   413-417, 1999.

5. Prosthetic Control by an EEG-based Brain-Computer Interface (BCI)
   Christoph Guger, Werner Harkam, Carin Hertnaes, Gert Pfurtscheller
   .Institute of Biomedical Engineering, Department of Medical
   Informatics 2Ludwig-Boltzmann Institute for Medical Informatics and
   Neuroinformatics .University of Technology Graz.

6. Cilliers & VanDerKouwe 1993 "A VEP-based Computer Interface for
   C2-Quadriplegics" Cilliers, P.J., and Van Der Kouwe, A.J.W., presented
   at 1993 IEEE Conference on Electronic Devices for the Disabled --
   Beyond 2000, Fall 1993.
				   
7. The Brain-Computer Interface: Techniques for Controlling Machines.
   Richard H.C. Seabrook

8. Suzushi Nishimura Yutaka Tomita and Toshio Horiuchi. Clinical
   application of an ative electrode using an operational
   amplifier. IEEE transactions on biomedical engineering, bme -30(1)
   January 1984.

9. Ramon Pallas-Areny, Joseph Colominas, and Javier Rosell. An
   Improved Buffer for Bioelectric signals. IEEE transactions on
   biomedical engineering, vol 36. No 4 Aprip p 490-493.

Contact Information 

M.J Meintjes
P.O Box 14484
Hatfield 
0028

Email   : thys@netsys.co.za
Work no.: (012) 348 4246
Work fax: (012) 348 4802
Cell no.: 082 3764 602

\end{verbatim}
\end{document}